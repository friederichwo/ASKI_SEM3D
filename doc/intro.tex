%
\section*{Introduction}
%
 This document explains how to do a hybrid full waveform inversion of teleseismic body waves using a combination of the codes GEMINI, SPECFEM3D-Cartesian and ASKI. To implement the hybrid approach, we define a regional computational box within which we perform SPECFEM3D-Cartesian calculations while the incident teleseismic wavefield is computed using GEMINI that is restricted to 1D earth models. In spite of the fact that we consider waves propagating in a spherical earth, we use the Cartesian version of SPECFEM3D. This is accomplished by applying a special mapping of the SPECFEM mesh to a spherical chunk. The GEMINI wavefields are injected at the boundaries of the SPECFEM box and continued using SPECFEM3D. For inversion using ASKI, we compute sensitivity kernels obtained from the displacement and strain fields available on the SPECFEM3D computational mesh. SPECFEM3D makes available these wavefields to ASKI which performs one iteration of the inversion. The new earth model after one iteration is fed back to SPECFEM3D for performing another forward calculation for the next iteration.

 In the following, we describe how GEMINI, SPECFEM3D and ASKI are set up and used to realize this workflow. Section 1 explains how the full waveform inversion is organized. In section 2, we start with the preparation of SPECFEM3D computations up to generating the parallel databases and inplementing a 1D background model. Then, in section 3, we explain how to use GEMINI to compute injection wavefields at the boundaries of the SPECFEM box. In section 4, it is described how to process the raw seismogram data to finally transform them into complex Fourier amplitudes evaluated at predefined frequencies. This section also explains how moment rate functions are estimated from the observed seimograms. Section 5 describes the setup of ASKI and how it supports running SPECFEM to get the required displacement and stress fields. Section 6 brings everything together and describes the computation of sensitivity kernels from the wavefields computed by SPECFEM, the extraction and transformation of synthetic seismograms computed by SPECFEM and finally the setup of the equation system to be solved for obtaining an update of the earth model.
